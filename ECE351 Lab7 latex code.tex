\documentclass[11pt]{article}
\usepackage{amsmath, amssymb, amscd, amsthm, amsfonts}
\usepackage{graphicx}
\usepackage{pdfpages}
\usepackage{hyperref}
\usepackage{float}
\usepackage{amsmath}
\oddsidemargin 0pt
\evensidemargin 0pt
\marginparwidth 40pt
\marginparsep 10pt
\topmargin -20pt
\headsep 10pt
\textheight 8.7in
\textwidth 6.65in
\linespread{1.2}
\usepackage[latin1]{inputenc}
\title{Lab 7}
\author{ECE 351-52}
\author{Meshari Alhajeri}
\date{10/21/2021}

\usepackage{listings}
\usepackage{xcolor}

\definecolor{codegreen}{rgb}{0,0.6,0}
\definecolor{codegray}{rgb}{0.5,0.5,0.5}
\definecolor{codepurple}{rgb}{0.58,0,0.82}
\definecolor{backcolour}{rgb}{0.95,0.95,0.92}

\lstdefinestyle{mystyle}{
backgroundcolor=\color{backcolour}, 
commentstyle=\color{codegreen},
keywordstyle=\color{magenta},
numberstyle=\tiny\color{codegray},
stringstyle=\color{codepurple},
basicstyle=\ttfamily\footnotesize,
breakatwhitespace=false, 
breaklines=true, 
captionpos=b, 
keepspaces=true, 
numbers=left, 
numbersep=5pt, 
showspaces=false, 
showstringspaces=false,
showtabs=false, 
tabsize=2
}


\usepackage{steinmetz}
\lstset{style=mystyle}
\newtheorem{theorem}{Theorem}
\newtheorem{lemma}[theorem]{Lemma}
\newtheorem{conjecture}[theorem]{Conjecture}

\newcommand{\rr}{\mathbb{R}}

\newcommand{\al}{\alpha}
\DeclareMathOperator{\conv}{conv}
\DeclareMathOperator{\aff}{aff}

\begin{document}

\maketitle
\section{Introduction :}
In this lab we learned that how we can use the scipy builtin functions to perform the partial factorization and expansion of the large complicated terms.
\section{Equations :}


\textbf{Task 1 part 1 :}

\begin{align*}
\\ G(s) = \frac{\left(s+9\right)}{\left(s^2-6s-16\right)\left(s+4\right)} 
\\ G(s) = \frac{s+9}{\left(s+2\right)\left(s-8\right)\left(s+4\right)}
\\A(s) = \frac{\left(s+4\right)}{\left(s^2+4s+3\right)}
\\A(s) = \frac{s+4}{\left(s+1\right)\left(s+3\right)}
\\B(s) = s^2+26s+168
\\B(s) = \left(s+12\right)\left(s+14\right)
\end{align*}

\textbf{Task 1 part 3 :}

\begin{align*}
Z(s) = A(s).X(s)
\\ Y(s) = G(s).Z(s)
\\ Y(s) = G(s).A(s).X(s)
\\ \frac{Y(s)}{X(s)} = \frac{s+9}{\left(s+2\right)\left(s-8\right)\left(s+4\right)} \times \frac{s+4}{\left(s+1\right)\left(s+3\right)}
\\ \frac{Y(s)}{X(s)} =\frac{s+9}{\left(s+1\right)\left(s+3\right)\left(s+2\right)\left(s-8\right)}
\end{align*}

\textbf{Task 1 Part 4 :}
\\ As, all of the poles of the transfer function are not negative , in case of s-8, pole is +8. So, it will produce a $e^{-8}$ term which will cause system to become very unstable.

\textbf{Task 1 Part 6 :}
\\ Plot matches the expected result as time passes the response of the system increases exponentially and grows very large in small amount of time.



\textbf{Task 1 part 3 :}

\begin{align*}
Z(s) &= A(s).X(s)
\\ Y(s) &= \frac{G(s)}{1+G(s)\cdot B(s)}.Z(s)
\\ Y(s) &= \frac{G(s)}{1+G(s)\cdot B(s)}.A(s)\cdot X(s)
\\ \frac{Y(s)}{X(s)} &= \frac{G(s)\cdot A(s)}{1+G(s)\cdot B(s)}.
\\\frac{Y(s)}{X(s)} &= \frac{\left(\left(\:\frac{s+9}{\left(s+2\right)\left(s-8\right)\left(s+4\right)}\right)\left(\:\frac{s+4}{\left(s+1\right)\left(s+3\right)}\right)\right)}{1+\left(\left(\:\frac{s+9}{\left(s+2\right)\left(s-8\right)\left(s+4\right)}\right)\left(\:\left(s+12\right)\left(s+14\right)\right)\right)}
\\ \frac{Y(s)}{X(s)} &= \frac{\left(s+9\right)\left(s+4\right)}{\left(s+1\right)\left(s+3\right)\left(2s^3+33s^2+362s+1448\right)}
\end{align*}

\textbf{Task2 part 3 :}
\\ As all of the poles of the transfer function are negative that means the function will gradually stabilize to a certain value as time passes.

\textbf{Task2 part 5 :}
\\ As expected, it is evident fro the graph that system response converges to 0.8 after some time and become stable.

\section{Results :}
\textbf{Task1 part 2 :}

\begin{verbatim}
For G(x)
Value of z = [-9.]
Value of p = [ 8. -4. -2.]
Value of K = 1.0

For A(x)
Value of z = [-4.]
Value of p = [-3. -1.]
Value of K = 1.0

For B(x)
Value of z = []
Value of p = []
Value of K = [ 1. 26. 168.]
\end{verbatim}



\textbf{Task1 part 4 :}
\begin{figure}[H]
\centering
\includegraphics[scale = 0.6]{Figure_1.png}
\caption{Step response of the }
\label{fig:my_label}
\end{figure}




\textbf{Task2 part 2 :}

\begin{verbatim}
Transfer function of the signal is = TransferFunctionContinuous( array([ 0.5, 6.5, 18. ]), array([1.0000e+00, 2.0500e+01, 2.5000e+02, 1.4975e+03, 3.4390e+03, 2.1720e+03]), )
\end{verbatim}



\textbf{Task1 part 4 :}
\begin{figure}[H]
\centering
\includegraphics[scale = 0.6]{Figure_2.png}
\caption{Step response of the }
\label{fig:my_label}
\end{figure}













\section{Questions :}
\textbf{Answer 1:}
\\np.convolve() takes discrete time convolution of two arrays, our user defined function also takes discrete time convolution of two arrays so their output should be same.


\textbf{Answer 2:}
\\open loop has no feedback mechanism so the output is not controlled, but in case of closed loop control system the system has a feedback mechanism which constantly adjust the error in the response of the function which makes it a stable system.

\textbf{Answer 3 :}
\\ Scipy.signal.residue gives partial fraction expansion of the fraction, while Scipy.signal.tf2zpk gives poles of the transfer function. 


\textbf{Answer 4 :}
\\ Open loop system can be stable in some situation especially in the simple system with simple input because they don't need much of feedback response in those case.
\\ Closed loop system can be unstable in some cases, the presence of feedback can cause the closed-loop system to become unstable, as in the case of higher order
.

{\subsection{Question no 5 :} }
\textbf{Answer :}
\\Lab was well explained and understandable.
\section{Conclusion :}
\begin{itemize}
\item In this lab we learned the difference between the closed loop and open loop control system .
\item I learned how to check for the stability of the function.
\item How to check the step response of the function using scipy signal API.
\item I learned how to perform polynomial multiplication use convolve() method to convolve two arrays. 
\end{itemize}
\end{document}
