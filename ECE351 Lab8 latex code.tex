\documentclass{article}
\usepackage[utf8]{inputenc}
\usepackage{graphicx}
\usepackage{float}
\title{Lab 8}
\author{ECE 351-52}
\author{Meshari Alhajeri}
\date{10/26/2021}

\begin{document}

\maketitle

\section{Introduction:}
In this lab we will learn how can we use python to find different values of $a_k$ , $b_k$. and how to add all these different equations to get the desired output.
\section{Equations:}
\subsection{Task 1}
As the given square is a odd function so it will have only $b_k$ values it's $a_0$ and $a_k$ value will be zero.
$$a_0 = 0$$
$$a_k = 0$$
$$b_k = \frac{2}{k \pi}\left(1- \cos (k \pi) \right)$$
$$x(t) = \frac{1}{2}a_0 \sum_{k=1}^{N}\left( a_k \cos(\omega_0 t ) + b_k \sin(\omega_0 t ) \right)$$
$$x(t) = \frac{2}{k \pi}\left(1- \cos (k \pi \right) \sin(\omega_0 t ) $$

\section{Results: }
\subsection{Task 2:}
\begin{figure}[H]
\centering
\includegraphics[scale=0.7]{Figure_1.png}

\end{figure}

\begin{figure}[H]
\centering
\includegraphics[scale=0.7]{Figure_2.png}

\end{figure}

\section{Questions :}
\textbf{Question no 1 :}
\\$x(t)$ is a odd function because it is the inverted version of the function function on the left side of the axis, also, because it's $a_0$ and $a_k$ is zero.
\\ \textbf{Question no 2 :}
\\Value of the $a_1,a_2,a_3$ is equal to zero because the function is the odd function and the value of the $a_0$ and $a_k$ is zero.
\\ \textbf{Question no 3 :}
\\As the value of the n increases the shape of the plotted function becomes more and more alike to the square function.
\\ \textbf{Question no 4 :}
\\ As, the no of n increases no of sine waves increases thus making it more similar to the desired function.
\\ \textbf{Question no 5 :}
\\ Lab manual was well explained and easily comprehend able.

\section{Conclusion :}
\begin{itemize}
\item In this lab we learned how we can use Fourier series to implement any function.
\item We learned how to take n summation of the Fourier series and what difference does occur by changing no of iterations.
\item how to use python to iteratively sum up terms for Fourier transform.
\end{itemize} 
\section{Appendix :}
\begin{verbatim}
Value of b for the first a0 is 0, a1 is 0, three bk is [0.00000000e+00 1.27272676e+00 1.01750234e-06]
\end{verbatim}

\end{document}