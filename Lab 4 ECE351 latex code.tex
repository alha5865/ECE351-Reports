\documentclass[11pt]{article}
\usepackage{amsmath, amssymb, amscd, amsthm, amsfonts}
\usepackage{graphicx}
\usepackage{pdfpages}
\usepackage{hyperref}
\usepackage{amsmath}
\oddsidemargin 0pt
\evensidemargin 0pt
\marginparwidth 40pt
\marginparsep 10pt
\topmargin -20pt
\headsep 10pt
\textheight 8.7in
\textwidth 6.65in
\linespread{1.2}

\title{Lab 4}
\author{ECE 351-52}
\author{Meshari Alhajeri}
\date{9/30/2021}

\usepackage{listings}
\usepackage{xcolor}

\definecolor{codegreen}{rgb}{0,0.6,0}
\definecolor{codegray}{rgb}{0.5,0.5,0.5}
\definecolor{codepurple}{rgb}{0.58,0,0.82}
\definecolor{backcolour}{rgb}{0.95,0.95,0.92}

\lstdefinestyle{mystyle}{
backgroundcolor=\color{backcolour}, 
commentstyle=\color{codegreen},
keywordstyle=\color{magenta},
numberstyle=\tiny\color{codegray},
stringstyle=\color{codepurple},
basicstyle=\ttfamily\footnotesize,
breakatwhitespace=false, 
breaklines=true, 
captionpos=b, 
keepspaces=true, 
numbers=left, 
numbersep=5pt, 
showspaces=false, 
showstringspaces=false,
showtabs=false, 
tabsize=2
}

\lstset{style=mystyle}
\newtheorem{theorem}{Theorem}
\newtheorem{lemma}[theorem]{Lemma}
\newtheorem{conjecture}[theorem]{Conjecture}

\newcommand{\rr}{\mathbb{R}}

\newcommand{\al}{\alpha}
\DeclareMathOperator{\conv}{conv}
\DeclareMathOperator{\aff}{aff}

\begin{document}

\maketitle



\section{Introduction}\label{section-introduction}

The purpose of this lab was to familiarize students with the step response calculation using python. In this lab we used lab 3 convolution function to find step response of function, first the functions were made using user defined functions and step function from lab 2.% If we denote by $\conv(S)$ the convex hull of a set $S \subset \rr^d$, it says the following

\section{Methodology :}\label{section-introduction}

\subsection{Part 1 :}
{\bf Task 1 : }


\begin{lstlisting}[language=Python, caption=Part 1(Task 1)]
def f1(time,signal) :
stepfunc(time,signal,0, 1)
stepfunc(time,signal,3, -1)
signal_ = np.exp(-2*time)
signal = np.multiply(signal_,signal)
return signal
def f2(time,signal) :
stepfunc(time,signal,2, 1)
stepfunc(time,signal,6, -1)
# signal_ = np.exp(-time)
# signal = np.multiply(signal_,signal)
return signal
def f3(time,signal) :
ramp_func(time,signal,2,1)
f = 2*0.25*22/7
signal = np.cos(f*signal)
return signal
\end{lstlisting}

{\bf Explanation : }
\\In this part of the code, we developed the user defined functions for creating the signals defined in the lab report, to create these functions I used the step function from lab 3 and additionally I used exp() function from Numpy module of python. 
In the task 2 we used the pre-defined functions from task 1 to plot the signals genrated, the xlmimt of the plot was set to $-10 \geq x \leq 10.$

\section{Results : }

\subsection{Part 1 :}
{\bf Task 2}

\centerline{\includegraphics[scale=0.7]{01.png}}
\centerline{caption{Plots of the functions}}
\label{fig:my_label}
\subsection{Part 2 :}
{\bf Task 2}



\centerline{\includegraphics[scale=0.7]{02.png}}
\centerline{caption{code generated response}}
\label{fig:my_label}

{\bf Task 2(Hand written solution)}

\centerline{\includegraphics[scale=0.7]{03.png}}
\centerline{caption{step response calculated by hand}}
\label{fig:my_label} 





\section{Questions : }

{\bf Feedback :}
\\The lab tasks were properly explained and the deliverable section is very helpful to know that what should be included in the lab report.


\section{Conclusion : }
In this lab we learned how to find unit step response of a function using convolution, we also compared the solution with the hand calculated solution.

\end{document}
