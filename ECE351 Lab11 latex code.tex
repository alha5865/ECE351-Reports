\documentclass{article}
\usepackage[utf8]{inputenc}
\usepackage{graphicx}
\usepackage{float}
\usepackage{amsmath}
\usepackage{amssymb}
\title{Lab 11}
\author{ECE 351-52}
\author{Meshari Alhajeri}
\date{10/16/2021}

\begin{document}

\maketitle

\section{Introduction}
In this lab we will learn how to find Z-transform of a system, how to find system equation by taking inverse Z transform, how t find and plot the poles and zeros of the system. How to use the freqz function to plot the magnitude and phase bode plot. 




\section{Methodology :}
\textbf{Task 1 :}
$$ y[k] = 2x[k]-40x[k-1]+10y[k-1]-16[k-1]-16y[k-2] $$
Now taking Z transform of the whole equation :
\begin{align*}
Y(z) & = 2X(z)-40X(z)z^{-1}+10Y(z)z^{-1}-16Y(z)z^{-2}
\\ Y(z)\{1-10z^{-1}+16z^{-2} \} & = X(z)\{2-40z^{-1} \}
\frac{Y(z)}{X(z)} &=
\end{align*}

Multiplying both sides of the equation by $x^2$
\begin{align*}
\frac{Y(z)}{X(z)} &=\frac{2z^{2}-40z}{z^2-10z+16}
\end{align*}

\textbf{Task 2 :}

\begin{align*}
\frac{2z^2-40z}{z^2-10z+16} &= 2+\frac{-20z-32}{z^2-10z+16}
\\ 2+\frac{-20z-32}{z^2-10z+16} &= 2+\frac{12}{z-2}-\frac{32}{z-8}
\end{align*}
Now taking inverse Z-transform :
$$3\times 2^{2n} - 4\times 2^{3n} $$










\section{Results}

\textbf{Task 4 :}
\begin{figure}[H]
\centering
\includegraphics[width = 300 px]{01.png}
\label{fig:my_label}
\end{figure}
\textbf{Task 5 :}
\begin{figure}[H]
\centering
\includegraphics[width = 300 px]{02.png}
\label{fig:my_label}
\end{figure}



.

\section{Appendix :}
\textbf{Printed output from Task 3 :}
\begin{verbatim}
The result of the residuez function is : [ 6. -4.], [2. 8.], []
\end{verbatim}

\section{Conclusion :}
In this lab we learned :
\begin{itemize}
\item how to find Z transform of a system.
\item how to do partial fraction and then find inverse Z- transform to find the signal in time domain.
\item Plot the magnitude and phase plot of the system.
\item Plot the poles and the zeros of the system on the graph.
\end{itemize}

\section{Questions :}
\textbf{Answer 1 :}
Looking at the plot of the pole and the zeros all the poles are on the left so the system is unstable.

\end{document}
