\documentclass{article}
\usepackage[utf8]{inputenc}
\usepackage{graphicx}
\usepackage{float}
\title{Lab 10}
\author{ECE 351-52}
\author{Meshari Alhajeri}
\date{11/11/2021}

\begin{document}

\maketitle

\section{Introduction}
In this lab we will learn how to plot transfer function of a circuit using its magnitude and phase plots.We will also learn how can we find response of circuit using the python signal API
\section{Results}

\textbf{Task 1 :}
\begin{figure}[H]
\centering
\includegraphics[scale = 0.6]{01.png}
\label{fig:my_label}
\end{figure}
\begin{figure}[H]
\centering
\includegraphics[scale = 0.6]{02.png}
\label{fig:my_label}
\end{figure}


\textbf{Task 2 :}
\begin{figure}[H]
\centering
\includegraphics[scale = 0.6]{03.png}
\label{fig:my_label}
\end{figure}
\begin{figure}[H]
\centering
\includegraphics[scale = 0.6]{07.png}
\label{fig:my_label}
\end{figure}

\textbf{Task 3 :}
\begin{figure}[H]
\centering
\includegraphics[scale = 0.6]{04.png}
\label{fig:my_label}
\end{figure}
\section*{Part 2 :}
\textbf{Task 1 :}
\begin{figure}[H]
\centering
\includegraphics[scale = 0.6]{05.png}
\label{fig:my_label}
\end{figure}

\textbf{Task 2 :}
\begin{figure}[H]
\centering
\includegraphics[scale = 0.6]{06.png}
\label{fig:my_label}
\end{figure}
.

\section{Conclusion :}
In this lab we learned :
\begin{itemize}
\item how to plot transfer function calculated by hand
\item how to find z transform of a function using scipy
\item how to plot the transfer function using bode plot
\end{itemize}
\section{Questions :}
\textbf{question 1 :}
\\ Filtered output makes perfect sense because the transfer function of the circuit is a band pass filter which allows only frequencies of mid range. And filters out the frequencies of higher and lower range. Same is the case with our output.
\textbf{question 2 :}
\\ We use scipy.signal.bilinear to convert transfer function to z domain and then we use lfilter to filter the output.
\textbf{question 3 :}
\\ If the sampling and signal frequency is different than the filtered signal is distorted.


\end{document}